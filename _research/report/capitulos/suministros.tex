\chapter{Cadenas de Suministros}\label{cap:supply}

La \iem{logística} de un sistema consiste en satisfacer ciertas
demandas, de materias primas, de tiempo, bajo ciertas restricciones,
normalmente presupuesto, mientras se intenta optimizar una medida de
desempeño, normalmente el costo de materiales y transporte.

En el ámbito industrial, la logística se denomina \iem{cadena de
  suministros}.  Esta tiene dos etapas, la \iem{cadena de producción}
y la \iem{cadena de distribución}\cite{Musmanno2004}.

\section{Formalización}

Su estructura depende del caso en particular, pero siguiendo el modelo
simplificado planteado por\cite{Canales2016}, se considera que la
cadena de producción tiene entrada fija de proveedores y plantas de
ensamblaje. A su vez, la cadena de distribución tiene centros de
distribución, proveídos por las plantas de ensamblaje, y puntos de
venta. Cada instalación tiene un costo de operación fijo. El
transporte entre cada instalación tiene un costo unitario.  Cada
instalación tiene una capacidad de producción máxima, y los puntos de
venta tiene una demanda. Formalmente el problema tiene

\begin{itemize}
\item Proveedores $S$, plantas de ensamblaje $P$, centros de
  distribución $D$ y puntos de venta $C$. Las instalaciones se
  denominan $I = S \cup P \cup D \cup C$.
\item Matriz de costo unitario $U \in \mathbb{M}_{|I| \times |I|}$,
  donde $U_{ij}$ representa el costo de mover una unida de productos
  de la instalación $i$ a la instalación $j$
\item Cada instalación tiene un costo fijo $F: I \to \mathbb{R}^{+}$.
\item Cada instalación tiene una capacidad (o demanda) fija
  $W: I \to \mathbb{R}^{+}$.
\item $P_{max}$ y $D_{max}$ para limitar la cantidad de plantas y
  centros de distribución a abrir.
\end{itemize}

El objetivo de la logística es encontrar como repartir la carga entre
las plantas de producción y en los centros de distribución para
respetar las restricciones y minimizando el costo. Formalmente

\begin{itemize}
\item Determinar $O: I \to \set{0, 1}$ que indique que instalaciones
  hay que abrir. Tanto los proveedores como los puntos de venta tiene
  que estar abiertos siempre.
\item Determinar matriz de cargas $X$, donde $X_{ij}$ es la cantidad
  de productos que va a circular entre la instalación $i$ y $j$.
\end{itemize}

\section{Programación lineal}

Dado una $O$ fija, encontrar $X$ se puede formular como un problema de
programación lineal, por lo que se pueden encontrar los valores
óptimos en tiempo polinomial. La formulación sería minimizar

\begin{equation}
  \label{eq:supply-cost}
  z(O) = \sum_{i, j \in I^{2}}{O(i)O(j)X_{i, j}U_{i, j}} +
  \sum_{i \in P \cup D}{O(i)F(i)}
  \text{ costo fijo + costo unitario }
\end{equation}

sujeto a

\begin{align*}
  \label{eq:supply-constraints}
  &\sum_{j \in I}{X_{i, j}} \leq W(i), \forall i \in I
    \text{ respetar capacidades }\\
  &\sum_{i \in I}{X_{i, c}} \geq W(c), \forall c \in C
    \text{ satisfacer demandas }\\
  &\sum_{i \in I}{X_{i, j}} = \sum_{k \in I}{X_{j, k}}, \forall j \in P \cup D
    \text{ flujo constante }\\
  &\sum_{p \in P}{O(p)} \leq P_{max}, \sum_{d \in D}{O(d)} \leq D_{max}
    \text{ respetar máximo de instalaciones}
\end{align*}

Luego, para resolver este problema usando el método Simplex (ver\label{app:linprog})
hay que pasar esta formulación a un formato equacional.
