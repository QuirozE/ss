\chapter{Introducción}
\label{cap:intro}

\section{Motivación}

Gran cantidad de problemas como simulación del clima, análisis de genomas,
investigación en nuevas energía o aprendizaje de máquina \cite{Pacheco2011}
requieren gran poder computacional para ser resueltos. Por lo que mayor poder
de cómputo disponible aumenta la cantidad de problemas tratables. En el pasado,
esto se lograba aumentando la velocidad y densidad de transistores en los
procesadores. Sin embargo, esta técnica está llegando a un límite físico.

Así que en los últimos años, para obtener un aumento en el rendimiento se usan
sistemas paralelos. Estos pueden variar desde un procesador con varios núcleos
hasta redes de computadores. El tener varias unidades de cómputo permite
obtener mejoras de rendimiento a pesar de no tener procesadores más veloces.

Pero hacerlo introduce problemas concernientes a la coordinación de todas las
partes. En este trabajo se hará el análisis necesario para paralelizar la
solución a un problema particular, prestando atención a que las técnicas usadas
pueda ser útiles en problemas que requieran otra infraestructura.

\section{Metas}

Existe una gran variedad de técnicas de optimización paralelizables, varios
métodos de paralelización, cada uno con ventajas y desventajas, y aún más
variedad de problemas que podrían beneficiarse de esas optimización. Por
restricciones externas, en este trabajo se considera la heurística de
Optimización por Enjambre de Partículas, paralelización por medio de GPUs y
como problema de estudio la logística de cadenas de suministros.

Aún así, la meta principal de este trabajo es que el proceso documentado aquí
sea útil para replicar el resultado en problemas similares.

Las contribuciones de este trabajo son:

\section{Objetivo}

El objetivo general de este trabajo consiste en analizar que tanto se puede
mejorar el rendimiento de una heurística de optimización dada por medio de
paralelización y analizar su desempeño en un problema en particular.

Los objetivos particulares consisten en

\begin{itemize}
  \item Implementar secuencialmente la heurística
  \item Determinar los puntos de paralelización de la técnica existente
  \item Implementar la heurística de forma paralela
  \item Analizar las mejoras en tiempos de la implementación paralela
  \item Analizar las mejoras en calidad de la implementación paralela
\end{itemize}

\section {Organización}

Este trabajo se desarrolla en seis capítulos, incluyendo esta introducción, y
una sección adicional de apéndices que complementan la investigación.

En \cref{cap:paralelism} se presentan los diferentes tipos de paralelismo,
así como sus ventajas, restricciones y problemáticas. Además se estudian
algunas técnicas comunes para paralelizar algoritmos secuenciales. En
\cref{cap:bpso} se presenta la heurística a paralelizar y se usan los criterios
presentados en \cref{cap:paralelism} para proponer una versión paralelizada.
En \cref{cap:supply}, se presenta el problema de cadenas de suministros. Se
incluye su formalización como problema de optimización, se mencionan las
dificultades para obtener soluciones óptimas y las técnicas usadas para
resolverlo, incluyendo BPSO. En \cref{cap:results} se realiza una comparación
cualitativa entre los recursos y la calidad de las soluciones tanto de la
solución secuencial como la solución paralela. Se concluye el trabajo con
el \cref{cap:conc}, donde se exponen los comentarios finales y se sugiere el
trabajo a realizar en próximas investigaciones.

Se anexan además varios apéndices. En el \cref{app:gpus} se describe la
arquitectura de las tarjetas CUDA de Nvidia, así como conceptos básicos de
CUDA y \texttt{nvcc}. El \cref{app:julia} es una guía de alto nivel del
lenguaje Julia, así como algunas bibliotecas útiles. En el \cref{app:linprog}
se da un breve repaso de programas lineales y del método Simplex.
