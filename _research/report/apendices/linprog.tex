\chapter{Programación Lineal}\label{app:linprog}

Un problema de optimización lineal requiere maximizar (o minimizar) una función
lineal $z$ bajo $m$ restricciones lineales.

\begin{align}
  a_{j} \le k_{j}\\
  a_{j} \ge k_{j}\\
  a_{j} = k_{j}
\end{align}

donde tanto $z$ como $a_{i}$ son combinaciones lineales de un conjunto
${x_{i}}_{i=1}^{k}$ de variables y $k_{j}$ son números reales. Esto puede ser
restrictivo en varios casos, pero aún así existen una gran cantidad de problemas
que pueden ser formulados de esta forma. Usando la linealidad de la función
objetivo y de las restricciones, se han desarrollado varios métodos para
encontrar soluciones óptimas en tiempo razonable.

\section{Forma ecuacional}

Un problema está en forma ecuacional si todas sus restricciones (excepto las de
no negativiad) son ecuaciones (i.e. $a_{j} = k_{j}$) con $k_{j} \ge 0$ y todas
las variables son no negativas. Para todo problema, es posible construir un
problema en forma ecuacional con la misma solución óptima. Para hacer esto se
puede hacer lo siguiente

\begin{itemize}
  \item Si $k_{j} < 0$, se multiplica la restricción por $-1$ y se usa alguno de
        los casos siguientes.
  \item Si se tiene una variable $x_{j}$ que puede tener valores positivos y
        nagativos, se usan dos nuevas variables para obtener la ecuación
        $x_{j} = x_{j}^{+} - x_{j}^{-}$, con $x_{j}^{-}, x_{j}^{+} \ge 0$.
  \item Para las desigualdades de la forma $a_{j} \le k_{j}$, se introduce una
        variable $s_{j}$ de \emph{sobra} (\emph{slack} en inglés). La ecuación
        equivalente sería $a_{j} + s_{j} = k_{j}$ y además se agrega la
        restricción $s_{j} \ge 0$

  \item Para las desigualdades $a_{j} \ge k_{j}$, se hace algo análogo al caso
        anterior. La ecuación resultante es $a_{j} - s_{j} = k_{j}$ y $s_{j}
        \ge 0$.
\end{itemize}
