%%%%%%%%%%%%%%%%%%%%%%%%%%%%%%%%%%%%%%%%%%%%%%%%%%%%%%%%%%%%%%%%%%%%%%%%%%%%%%%%
\chapter{Paralelismo}

Gran cantidad de problemas como simulación del clima, análisis de genómas,
investigación en nuevas energía o aprendizaje de máquina \cite{Pacheco2011}
requieren gran poder computacional para ser resueltos. Por lo que mayor poder
de cómputo disponible aumenta la cantidad de problemas tratables. En el pasado,
esto se lograba aumentando la velocidad y densidad de transistores en los
procesadores. Sin embargo, esta técnica está llegando a un límite físico.

Así que en los últimos años, para obtener un aumento en el rendimiento se usan
sistemas paralelos. Estos pueden variar desde un procesador con varios núcleos
hasta redes de computadores. El tener varias unidades de cómputo permite obtener
mejoras de rendimiento a pesar de no tener procesadores más veloces.

Pero hacerlo introduce problemas concernientes a la coordinación de todas las
partes. Los detalles del problema y las diferentes técnicas y herramientas para
lidiar con ellos dependen bastante de la infraestructura particular.

Más aún, cierta infraestrucutura o técnica pueden ser más conveniente de usar
para ciertos tipos de problemas. En este capítulo se van a explorar diferentes
clasificaiones de arquitecturas paralelas y sus peculiaridades, además de dar
algunos criterios para clasificar los problemas y decidir la mejor manera de
crear una solución paralela a un problema en particular.

\section{Tipos de arquitectura}

Tradicionalmente se dividen los tipos de sistemas en base a como tratan el flujo
de datos y de instrucciones. Esto se conoce como taxonomía de Flynn. Están las
siguientes  categorías

\begin{itemize}
  \item \textit{SISD}\index{SISD} (Single Instruction Single Data): sistemas
    sin paralelismo.

  \item \textit{SIMD}\index{SIMD} (Single Instruction Multiple Data):
    sistemas que pueden aplicar una operación a vectores o arreglos de datos en
    una unidad de tiempo

  \item \textit{MIMD}\index{MIMD} (Multiple Data Multiple Instructions):
    sistemas donde cada unidad de procesamiento es independiente.
\end{itemize}

\subsection{Sistemas \textit{SISD}}

Contrario a su nombres, estos modelos pueden tener ciertos mecanismos
paralaelos. Estos son optimizaciones a nivel de hardware que permiten ejecutar
instrucciones de forma no secuencial \index{ILP}. Esto no se considera
paralelismo en la taxonomía de Flynn ya que estos mecanismos no están
disponibles para los desarrolladores. Aún así, vale la pena revisar algunos.

\begin{itemize}
  \item \emph{Pipelining}\index{pipelining}:
        Consiste en partir una instrucción en pasos atómicos. Esto es sacar los
        datos de memoria, revisar el tipo de operación, aplicar la operación,
        guardar los datos resultantes, entre otros. Cada sección se le asigna a
        diferentes secciones de la unidad de control. Entonces, cuando un
\end{itemize}

\index{multiple issue}

\subsection{Sistemas \textit{SIMD}}

En estos sistemas hay una única unidad de control con varias unidades aritmético
lógicas (\textit{ALU}). Al recibir la instrucción, la unidad de control avisa a
todas las \textit{ALUs} para que apliquen la operación al dato correspondiente.
Este tipo de paralelismo se denomina \textbf{paralelismo de datos}
\index{paralelismo de datos}.

Deben tener registros capaces de guarda vectores y operaciones optimizadas para
leer y escribir de los elementos en estos registros. Todas las operaciones son
síncronas y totalmente uniformes. Además de operaciones sobre los elementos,
suelen haber operaciones que actúan sobre los vectores sin tener que acceder a
cada elementos, como obtener la longitud del vector.

A pesar de solo ser factibles en operaciones muy restringidas, estos sistemas
no suelen introducir problemas nuevos a los programas y su uso es bastante
directo. Además, la mejora en rendimiento al usarlos suele ser aceptable.

Hoy en día, cada núcleo de un \textit{CPUs} \index{CPU} cuentan con
instrucciones que permiten este tipo de operaciones. No son un sistema
exclusivamente \textit{SIMD}, ya que tener varios núcleos le dan
características \textit{MIMD}.

Las unidades de procesamiento gráfico (\textit{GPU}) \index{GPU} también tienen
este tipo de funcionalidad. Además, como las imágenes suelen ocupar mucha
memoria, tiene memoria de gran tamaño optimizada para el manejo con hilos a
nivel de hardware \index{multiple issue}. Curiosamente, esto hace que tenga mal
desempeño en problemas pequeños. Al igual que los \textit{CPUs}, tampoco son
exclusivamente un sistema \textit{SIMD}, ya que suelen tener más de un núcleo.

\subsection{Sistemas \textit{MIMD}}

%%%%%%%%%%%%%%%%%%%%%%%%%%%%%%%%%%%%%%%%%%%%%%%%%%%%%%%%%%%%%%%%%%%%%%%%%%%%%%%%
\section{Tipos de Problemas}

Pancake \cite{Pancake1996} propone algunas reglas que a grandes rasgos permitan
determinar si vale la pena dar una solución paralela a un problema.

\begin{enumerate}

    \item Si el problema es de paralelismo perfecto, probablemente sea fácil de
    paralelizar y se obtengan ganancia considerable en rendimiento.

    \item Si el problema es de paralelismo de \textit{pipeline}, sólo valdría
    la pena paralelizarlo si se puede balancear la carga de trabajo entre las
    etapas.

    \item Si el problema es de paralelismo totalmente síncrono, valdría la pena
    paralelizar dependiendo de que tan uniformemente se puede distriubir la
    carga.

    \item Si el problema es de paralelismo vagamente síncrono, solo valdría la
    pena paralelizarlo si hay muy pocas interacciones entre procesos.

    \item Un problema de paralelismo perfecto podría implementarse en un sistema
        \textit{MIMD}, pero sería problemático hacerlo en un sistema \textit{SIMD}.

    \item Un problema de paralelismo de \textit{pipeline} probablemente se
    desempeñe mejor en un sistema de memoria compartida o en un SMP (siempre y
    cuando cada etapa queda en una unidad de procesamiento). Podría funcionar
    decentemente en un sistema distribuido si la conexión entre etapas es lo
    suficientemente rápida.
    
    \item Un problema totalmente síncrono probablemente se desempeñe mejor en un
    sistema SIMD. Pero esto requeriría que todas las operaciones sean uniformes.
    De no ser así, una opción decente sería un sistema de memoria compartida.
    
    \item Para un problema vagamente síncrono la mejor opción sería un sistema
    de memoria compartida. Un sistema distribuido funcionaría siempre y cuando
    haya muchas operaciones entre cada interacción de los procesos.
    
    \item Lenguaje
    
    \item El tiempo de ejecución de una versión secuencia del programa se puede
    usar para estimar el desempeño del programa paralelo.
    
    \item Debido a las restricciones de las secciones secuenciales del programa,
    probablemente no valga la pena paralelizar un programa con una fracción
    paralelizable menor a 0.95.
    
    \item Hay que estimarlo con cuidado como cambia el desempeño del programa
    al cambiar la entrada. Este depende de la naturaleza del problema.
    
    \item El desempeño final siempre será peor al estimado.
    
    \item Aunque se puede intentar resolver pérdidas de tiempo surgidas por la
    concurrencia, en general dependerá de la naturaleza del problema y del
    equipo usado.
    
    \item Un problema de baja granuralidad tendrá un desempeño decente en
    cualquier sistema. Un sistema de granuralidad media/alta probablemente solo
    sea decente en un sistema SIMD.
    
    \item Para sistemas distribuidos, es útil estimar la equivalencia de
    mensaje.
    
\end{enumerate}

%%%%%%%%%%%%%%%%%%%%%%%%%%%%%%%%%%%%%%%%%%%%%%%%%%%%%%%%%%%%%%%%%%%%%%%%%%%%%%%%
\section{Métricas de desempeño}

\section{Resumen}
