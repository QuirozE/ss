\chapter{Cadenas de Suministros}
\label{cap:supply}

La \iem{logística} de un sistema consiste en satisfacer ciertas demandas,
de materias primas, de tiempo, bajo ciertas restricciones, normalmente
presupuesto, mientras se intenta optimizar una medida de desempeño, normalmente
el costo de materiales y transporte.

En el ámbito industrial, la logística se denomina \iem{cadena de suministros}.
Esta tiene dos etapas, la \iem{cadena de producción} y la \iem{cadena de
distribución} \cite{Musmanno2004}.

\section{Formalización}

Su estrucutura depende del caso en particular, pero siguiendo el modelo
simplificado planteado por \cite{Canales2016}, se considera que la cadena de
producción tiene entrada fija de proveedores y plantas de ensamblaje. A su vez,
la cadena de distribución tiene centros de distribución, proveídos por las
plantas de ensamblaje, y puntos de venta. Cada instalación tiene un costo de
operación fijo. El transporte entre cada instalación tiene un costo unitario.
Cada instalación tiene una capacidad de producción máxima, y los puntos de venta
tiene una demanda. Formalmente el problema tiene

\begin{itemize}
  \item Proveedores $S$, plantas de ensamplaje $P$, centros de distribución $D$
        y puntos de venta $C$. Las instalaciones se denominan
        $I = S \cup P \cup D \cup C$.
  \item Matrix de costo unitario $U \in \mathbb{M}_{|I| \times |I|}$, donde
        $U_{ij}$ representa el costo de mover una unida de productos de la
        instalación $i$ a la instalación $j$
  \item Cada instalación tiene un costo fijo $F: I \to \mathbb{R}^{+}$.
  \item Cada instalación tiene una capacidad (o demanda) fija
        $W: I \to \mathbb{R}^{+}$.
  \item $P_{max}$ y $D_{max}$ para limitar la cantidad de plantas y centros de
        distribución a abrir.
\end{itemize}

El objectivo de la logística es encontrar como repartir la carga entre las
plantas de producción y en los centros de distribución para respetar las
restricciones y minimizando el costo. Formalmente

\begin{itemize}
  \item Determinar $O: I \to \set{0, 1}$ que indique que instalaciones
        hay que abrir. Tanto los proveedores como los puntos de venta tiene
        que estar abiertos siempre.
  \item Determinar matrix de cargas $X$, donde $X_{ij}$ es la cantidad de
        productos que va a circular entre la instalación $i$ y $j$.
\end{itemize}

\section{Programación lineal}

Dado una $O$ fija, encontrar $X$ se puede formular como un problema de
programación lineal, por lo que se pueden encontrar los valores óptimos
en tiempo polinomial. La formulación sería minimzar

\begin{equation}
  \label{eq:supply-cost}
  z(O) = \sum_{i, j \in I^{2}}{O(i)O(j)X_{i, j}U_{i, j}} +
  \sum_{i \in P \cup D}{O(i)F(i)}
  \text{ costo fijo + costo unitario }
\end{equation}

sujeto a

\begin{align*}
  \label{eq:supply-constraints}
  &\sum_{j \in I}{X_{i, j}} \leq W(i), \forall i \in I
    \text{ respetar capacidades }\\
  &\sum_{i \in I}{X_{i, c}} \geq W(c), \forall c \in C
    \text{ satisfacer demandas }\\
  &\sum_{i \in I}{X_{i, j}} = \sum_{k \in I}{X_{j, k}}, \forall j \in P \cup D
    \text{ flujo constante }\\
  &\sum_{p \in P}{O(p)} \leq P_{max}, \sum_{d \in D}{O(d)} \leq D_{max}
    \text{ respetar máximo de instalaciones}
\end{align*}

Luego, para resolver este problema usando el método Simplex (ver
\label{app:linprog}) hay que pasar esta formulación a un formato equacional.

\section{Enjambre de partículas}

Una de las técnicas presentadas por \cite{Canales2016} es una variante de la
técnica de optimización por enjambre de partículas (PSO por sus siglas en
inglés). Esta es una heuristica bioinspirada en los movimientos de parvadas
de pájaros, que carecen de lider pero imitando localemente a las aves cercanas
surgen patrones complejos.

La técnica original funciones con espacios continuos, en general
$\mathbb{R}^{n}$, así que la técina presentada por \cite{Canales2016} es una
variante para trabajar con espacios binarios. En esta técnica se tiene un
conjunto $X$ de $k$ partículas, donde cada una representa una solución al
problema de optimización.

Cada partícula tiene una velocidad $V$ asociada. En cada paso, la velocidad se
actualiza usando la \cref{eq:bpso-vup}.

\begin{equation}
  \label{eq:bpso-vup}
  v \gets \phi v + c_{1}U(0, 1)(x^{\star} - x)
  + c_{2}U(0, 1)(\mathbf{x}^{\star} - x)
\end{equation}

donde $x^{\star}$ es la mejor posición alcanzada por$x$, $\mathbf{x}^{\star}$ es
la mejor posición global actual, $\phi$ es la constante de aceleración, $c_{i}$
son constantes para determinar cuando influencia la mejor solución personal y
global.

Luego, usando la velocidad actualizada se puede actualizar el $i$ésimo
componente de la particula $x$ con la \cref{eq:bpso-xup}.

\begin{equation}
  \label{eq:bpso-xup}
  x_{i} \gets \begin{cases}
    0, \text{ si } U(0, 1) \leq f(v)\\
    1, \text{ en otro caso }
  \end{cases}
\end{equation}

donde $f(v) = \frac{1}{1+e^{-v}}$. Una secuencia general de los pasos de la
optimización se puede ver en el \cref{alg:bpso}.

\begin{algorithm}
  \caption{Enjambre de particulas binarias}
  \label{alg:bpso}
  \begin{algorithmic}[1]
    \Require{Función $z$ a optimizar, $n$ número de partículas,
      $k$ número de rondas, $\phi$ constante de aceleración, $c_{1}, c_{2}$
      constantes de influencia}
    \Ensure{Mejor solución $x$ encontrada}
    \Function{BPSO}{$z$, $k$, $n$}
      \State{Generar $n$ soluciones aleatorias y guardarlas en $X$}
      \State{Inicilizar velocidades $V$ en cero y mejores soluciones $X^{\star}$}
      \State{Encontrar mejor solución actual $x^{\star}$}
      \For{$t \in 0..k$}
        \For{$i \in 0..n$}
          \State{Actualizar $V[i]$ usando la \cref{eq:bpso-vup}}
          \State{Actualizar $X[i]$ usando la \cref{eq:bpso-xup}}
          \State{Aplicar optimización adicional}\label{alg:pso-opt}
          \Comment{Opcional}
          \State{Actualizar mejor solución global $x^{\star}$ y local $X^{\star}$}
        \EndFor
      \EndFor
      \Return{$x^{\star}$}
    \EndFunction
  \end{algorithmic}
\end{algorithm}

Esta técnica se puede usar para encontrar las instalaciones óptimas para abrir,
pues es un vector binario, con un flujo de productos fijo. Para optimizar ese
flujo, hay que notar que se tiene una $x$ fija en cada ciclo, así que se puede
construir la formulación de la \cref{eq:supply-cost} para optimizar los flujos
de productos.

Sería una optimización en dos partes. Con BPSO optimizando las instalaciones
a abrir y algún método de programación lineal para optimizar los flujos de
productos.
