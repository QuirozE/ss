\chapter{Introducción}
\label{cap:intro}

\section{Motivación}

Gran cantidad de problemas como simulación del clima, análisis de genómas,
investigación en nuevas energía o aprendizaje de máquina \cite{Pacheco2011}
requieren gran poder computacional para ser resueltos. Por lo que mayor poder
de cómputo disponible aumenta la cantidad de problemas tratables. En el pasado,
esto se lograba aumentando la velocidad y densidad de transistores en los
procesadores. Sin embargo, esta técnica está llegando a un límite físico.

Así que en los últimos años, para obtener un aumento en el rendimiento se usan
sistemas paralelos. Estos pueden variar desde un procesador con varios núcleos
hasta redes de computadores. El tener varias unidades de cómputo permite obtener
mejoras de rendimiento a pesar de no tener procesadores más veloces.

Pero hacerlo introduce problemas concernientes a la coordinación de todas las
partes. Los detalles del problema y las diferentes técnicas y herramientas para
lidiar con ellos dependen bastante de la infraestructura particular.

\section{Objetivo}

El objetivo general de este trabajo consiste en analizar que tanto se puede
mejorar el rendimiento de una heuristica existente para el problema de la
optimización de cadenas de suministros.

Los objetivos particulares consisten en

\begin{itemize}
  \item Determinar los puntos de paralelización de la técnica existente
  \item Analizar las mejoras en tiempos de la implementación paralela
  \item Analizar las mejoras en calidad de la implementación paralela
\end{itemize}

\section{Metas}
Se establecieron como metas:

Las contribuciones de este trabajo son:

\section {Organización de la tesis}

Este trabajo se desarrolla en seis capítulos, incluyendo esta introducción, y
una sección adicional de apéndices que complementan la investigación.

En \cref{cap:supply}, se presenta el problema de cadenas de suministros. Se
incluye su formalización como problema de optimización, se mencionan las
dificultades para obtener soluciones óptimas y las técnicas usadas para
resolverlo. En \cref{cap:paralelism} se presentan los diferentes tipos de
paralelismo y se estudian algunas técinas comunes para paralelizar algoritmos
secuenciales. En \cref{cap:parallel-bpso} se analiza la heuristica propuesta en
\cref{cap:supply} usando los criterios presentados en \cref{cap:paralelism} y
se propone una versioń paralelizada. En \cref{cap:results} se realiza una
comparación cualitativa entre los recursos y la calidad de las soluciones tanto
de la solución secuencial como la solución paralela. Se concluye el trabajo con
el \cref{cap:conc}, donde se exponen los comentarios finales y se sugiere el
trabajo a relizar en próximas investigaciones.

Se anexan además varios apéndices. En el \cref{app:gpus} se describe la
arquitectura de las tarjetas CUDA de Nvidia, así como conceptos básicos de
CUDA y \texttt{nvcc}. En el \cref{app:julia} se introducen conceptos básicos
del lenguaje Julia, así como algunas bibliotecas útiles. En el \cref{app:linprog}
se da un breve repaso de programas lineals y del método Simplex.
