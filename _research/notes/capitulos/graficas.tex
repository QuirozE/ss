%%%%%%%%%%%%%%%%%%%%%%%%%%%%%%%%%%%%%%%%%%%%%%%%%%%%%%%%%%%%%%%%%%%%%%%%%%%%%%%%
\chapter{Optimización sobre gráficas}

\section{Motivación}

\section{Distancia}

Intuitivamente, una trayectoria es una manera de llegar de un vértice a otro a
través de las aristas. Formalmente, un~\index{camino}camino es una gráfica tal
que existe un orden lineal de sus vértices

\[
    p = (x_{0}, x_{1}, \dots, x_{k-1}, x_{k})
\]

de tal manera que dos vértices son adyacentes si y solo si son consecutivos en
el orden. La~\index{longitud}longitud $l(p) = k$ del camino es la cantidad de
arista que tiene.

Una trayectoria es un camino que no repite vértices, y una $uv$-trayectoria es
una gráfica que sea trayectoria, que inicie en $u$ y que termine en $v$.

Es claro que entre un par de vértices puede haber más de una trayectoria. La
longitud de la más corta se denomia la~\index{distancia}distancia $d(u, v)$ de
los vértices.

Un 

Un 

%%%%%%%%%%%%%%%%%%%%%%%%%%%%%%%%%%%%%%%%%%%%%%%%%%%%%%%%%%%%%%%%%%%%%%%%%%%%%%%%
\subsection{Algoritmo de Dijkstra}

\subsection{Algoritmo de Bellman-Ford}

\section{Ciclos hamiltoneanos}

\subsection{Algoritmos genéticos}

\subsection{Enjambre de partículas}

\subsection{Colonia de hormigas}

\section{Flujos}

\subsection{Método de Ford-Fulkerson}

\section{Conjuntos independientes}

\section{Clanes}

\section{Resumen}
