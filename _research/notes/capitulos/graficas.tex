%%%%%%%%%%%%%%%%%%%%%%%%%%%%%%%%%%%%%%%%%%%%%%%%%%%%%%%%%%%%%%%%%%%%%%%%%%%%%%%%
\chapter{Optimización sobre gráficas}

Dado un conjunto de elementos de cualquier tipo, se puede abstraer las
relaciones entre ellos con diagramas. Cada elemento podría ser un punto que está
unido por una línea a los demás puntos con los que está relacionado. Estos
elementos podrían ser relaciones de amistad en una red social, vías de
comunicación entre centros de abastecimiento, o conxiones de red entre
servidores.

Estas abtracciones suelen ser suficientes para resolver problemas importantes
en estos contextos. Por ejemplo, saber la ruta más corta para distribuir
productos entre varios lugares, saber la mínima cantidad de servidores a retirar
para colapsar una red o saber como maximizar el flujo de energía en una red
eléctrica.

Muchos de estos problemas se pueden plantear como problemas de optimización.
Varias técnicas, tanto algorítimicas como heurísticas, se han desarrollado para
resolver estos tipos de problemas. A continuación se presentar algunas de ellas.

\section{Definiciones básicas}

Para los propósitos de este trabajo, se usarán las definiciones de
\cite{Diestel2017}. En estas, se toma una \iem{gráfica} $G$ como una tupla
$\tpl{V, E}$, donde $V$ es un conjunto no vacío y $E$ es una familia de
subconjuntos de $V$, todos de cardinalidad dos. Además, se pide que $V$ y $E$
sean disjuntos. Se suele denotar estos conjuntos como $V(G)$ y $E(G)$ para
evitar ambigüedad.

Los elementos de $V$ se denominan \iem{vértices}, y los de $E$ \iem{aristas}.
Para simplificar notación, una arista $\set{u, v}$ se suele denotar como $uv$.
La cantidad de vértices en una gráfica se llama el \iem{orden} y la cantidad de
aristas el \iem{tamaño}. Una arista $e$ es \iem{incidente} a un vértice $v$ si
$v \in e$. Si existe una arista $uv$, entonces $u, v$ son \iem{adyacentes}. El
conjunto de vértices adyacente a un vértice $v$ se denomia su \iem{vecindad} y
se denota como $N_{G}(v)$ o $N(v)$ si no hay ambigüedad.

Luego, sean $G, H$ dos gráfica, y $\phi: V(G) \to V(H)$ una función que preserva
adyacencias. Es decir, si $uv \in E(G)$, entonces $\phi(x)\phi(y) \in E(H)$.
$\phi$ es un \iem{homomorfismo}. Si $\phi^{-1}$ también es un homomorfismo,
entonces $\phi$ es \iem{isomorfismo}. En este caso se dice que $G$ y $H$ son
isomorfas y se denota como $G \cong H$. En general no se hace distinción entre
gráficas isomorfas.

Dadas dos gráficas $G, H$, si tanto los vértices como las aristas de $H$ son
subconjuntos de los vértices y aristas de $G$, $H$ es una \iem{subgráfica} de
$G$ y $G$ es una \iem{supergráfica} de $H$, que se denota como $H \subseteq G$.

\section{Distancia}

Intuitivamente, una trayectoria es una manera de llegar de un vértice a otro a
través de las aristas. Formalmente, una \iem{trayectoria} es una gráfica tal que
existe un orden lineal de sus vértices

\[
    p = (x_{0}, x_{1}, \dots, x_{k-1}, x_{k})
\]

de tal manera que dos vértices son adyacentes si y solo si son consecutivos en
el orden. La \iem{longitud} $l(p) = k$ del camino es la cantidad de arista en el
camino.

Una $uv$-trayectoria es una subgráfica que sea trayectoria, que inicie en $u$ y
que termine en $v$. Es claro que entre un par de vértices puede haber más de una
trayectoria. La longitud de la más corta se denomia la \iem{distancia} $d(u, v)$
de los vértices, que es infinito si no existe ninguna trayectoria. Un problema
básico en gráfica es que dada una gráfica $G$ y dos de sus vértices $u, v$,
encontrar su distancia.

\subsection{Sin pesos}

En su forma más simple, se puede considerar que todas las aristas representan
la misma distancia. Entonces, lo que se busca es la trayectoria con menos
aristas.

Esto se puede hacer iniciando en $u$ y revisando todos los vértices vecinos, que
serían los vértices a nivel uno. Luego, se revisan a los vértices en el nivel 2,
que serían los vecinos de los vecinos que no hayan sido revisados ya. Esto se
realiza sucesivamente hasta que no haya más vértices por revisar. Al final, el
nivel de $v$ corresponde a su distancia desde $u$. Esto no es tan obvio y se
justificará formalmente más adelante.

Este tipo de acciones se conocen como búsquedas, pues se busca cierta propiedad
de los vértices, en este caso el nivel de $v$. En particular, una búsqueda donde
se revisan todos los vértices a una nivel dado antes de avanzar se conoce como
una \iem{búsqueda en amplitud}, o BFS por sus siglas en inglés.

Para decidir el siguiente vértice a revisar se requiere algún tipo sala de
espera, donde se guarden los vértices vecinos al vértice actual que deberán ser
visitados eventualmente. Inductivamente, si se procesan los vértices en orden de
exploración, entonces todos los vértices del nivel $k$ estarían en la sala de
espera antes de que siquiera se explore alguno del nivel $k+1$.

Una estrucutura que respete el orden de llegada, FIFO por sus siglas en inglés,
se conoce como una \iem{cola}, y tiene dos operaciones. $push$ que añade un
elemento al final de la cola y $pop$ que saca el primer elemento, el más antigüo
, de la cola. Los detalles del proceso están descritos en el \cref{alg:bfs}.

\begin{algorithm}
  \caption[BFS]{Distancias usando búsqueda en amplitud(BFS)}
  \label{alg:bfs}
  \begin{algorithmic}[1]
    \Require{$G$ gráfica, $r$ vértice}
    \Ensure{Función de nivel $l$}
    \Function{bfs}{$G$, $r$}
      \State{$Q \gets$ []}
      \State \Call{push}{$Q$, $r$}
      \State{$l(r) = 0$}
      \While{$Q \ne$ []}
        \State{$x \gets$} \Call{pop}{$Q$}
        \ForAll{$v_{x} \in N(x)$}
          \If{$v_{x}$ no ha sido visitado} \label{alg:bfs:visit}
            \State{$l(v_{x}) = l(x) + 1$} \label{alg:bfs:level}
            \State{marcar $v_{x}$ como visitado}
            \State \Call{push}{$Q$, $v_{x}$}
          \EndIf
        \EndFor
      \EndWhile
      \Return{$l$}
    \EndFunction
  \end{algorithmic}
\end{algorithm}

Intuitivamente, si un vértice $v$ está a distancia $k$ de $r$, existe una
$rv$-trayectoria $p = (x_{0} = r, x_{1}, \dots, x_{k} = v)$ de logitud $k$. El
nivel de $r$ es cero. Luego, el nivel de $x_{1}$ es a lo más 1, pues es
adyacente a $r$. Y no puede ser cero, pues entonces existiría una
$rv$-trayectoria más corta, así que $l(x_{1}) = 1 = d(r, x_{1})$. Este argumento
se puede usar inductivamente para todos los vértices de $p$, y concluir que
$l(x_{i}) = i = d(r, x_{i})$, y en particular $l(v) = k = d(r, v)$. Formalmente,
esto se enuncia en el \cref{theo:bfs}.

\begin{theorem}
  \label{theo:bfs}
  La función de nivel $l(v)$ devuelta por el \cref{alg:bfs} corresponde a la
  distancia entre $r$ y $v$. Es decir, para todo vértice $v$, $l(v) = d(r, v)$.
\end{theorem}

\begin{proof}
  Sea $l(v) = bfs(G, r)$ la función de nivel y $v$ un vértice de $G$. Cada
  vértice en nivel $k + 1$ es adyacente a un vértice en el nivel $k$, por lo que
  para cualquier vértice, se puede construir una $rv$-trayectoria de longitud
  $l(v)$ siguiendo estas adyacencias, así que $l(v) \ge  d(r, v)$.

  Para la otra desiguladad, se prosigue por inducción sobre $d(r, v)$. Con
  $d(r, v) = 0$, $v$ es $r$ y $l(r) = 0$. Para $d(r, v) > 0$, sea $(x_{0} = r,
  \dots, x_{k} = v)$ una $rv$-trayectoria de longitud mínima. Por hipótesis de
  inducción, $d(r, x_{k-1}) \ge l(x_{k-1})$. Luego, como $x_{k-1}$ y $v$ son
  adyacentes, por la \cref{alg:bfs:visit}, hay dos posibilidades para $l(v)$.
  Primero, puede ser que $v$ ya haya sido visitado. Como las visitas se hacen
  por niveles, entonces el nivel del padre de $v$ es a lo más el nivel actual
  $l(x_{k-1})$. Así

  \[
    l(v) \le l(x_{k-1}) + 1 \le d(r, x_{k-1}) + 1 = d(r, v)
  \]

  En otro caso, $v$ no ha sido visita, así que por la \cref{alg:bfs:level} se
  tiene que

  \[
    l(v) = l(x_{k-1}) + 1 \le d(r, x_{k-1}) + 1 = d(r, v)
  \]

  En cualquier caso, se tiene que $l(v) \le d(r, v)$. Y como se tiene la
  desigualdad opuesta, se puede concluir que $l(v) = d(r, v)$.
  \end{proof}

Ahora, que pasaría si la distancia que representa cada arista no fuera uniforme.
En este caso, el algoritmo BFS no necesariamente daría la trayectoria más corta.
Este tipo de gráficas se llaman gráficas con pesos.

%%%%%%%%%%%%%%%%%%%%%%%%%%%%%%%%%%%%%%%%%%%%%%%%%%%%%%%%%%%%%%%%%%%%%%%%%%%%%%%%
\subsection{Con pesos}

Para extender la definición de \cite{Diestel2017}, se usarán conceptos de
\cite{Bondy2008}. Se define a una gráfica con pesos como una tupla $\tpl{G, w}$
donde $G$ es una gráfica y $w$ es una función que asigna pesos a las aristas
$w: E \to \mathbb{R}$. Por ahora, digamos que los pesos son positivos.

\begin{algorithm}
  \caption[Dijsktra]{
    Distancia con pesos usando el algoritmo de Dijkstra
  }
  \label{alg:dijkstra}
  \begin{algorithmic}[1]
    \Require{$G$ gráfica, $r$ vértice raíz}
    \Ensure{Función de distancia $l(v) = d_{G}(r, v)$}
  \end{algorithmic}
\end{algorithm}
%%%%%%%%%%%%%%%%%%%%%%%%%%%%%%%%%%%%%%%%%%%%%%%%%%%%%%%%%%%%%%%%%%%%%%%%%%%%%%%%
\subsection{Con pesos negativos}

%%%%%%%%%%%%%%%%%%%%%%%%%%%%%%%%%%%%%%%%%%%%%%%%%%%%%%%%%%%%%%%%%%%%%%%%%%%%%%%%
\section{Flujos}

\subsection{Método de Ford-Fulkerson}

\section{$\mathcal{NP}$-Completez}

\subsection{Ejemplos de problemas}

\subsubsection{Trayectorias hamiltoneánas}

\subsubsection{Clanes}

\subsubsection{Conjuntos independientes}

\subsection{Métodos evolutivos}

\subsubsection{Algoritmos genéticos}

\subsubsection{Enjambre de partículas}

\subsubsection{Colonia de hormigas}

\section{Resumen}
